\documentclass[eng,printmode,openany]{mgr}
\usepackage[utf8]{inputenc}
\usepackage{polski}
\usepackage[polish]{babel}
\usepackage{graphicx}
\usepackage{subfigure}

\usepackage{psfrag}
\usepackage{amsmath}
\usepackage{amsfonts}

\usepackage{supertabular}
\usepackage{array}
\usepackage{tabularx}
\usepackage{hhline}
\usepackage{showlabels}

\newcommand{\R}{I\!\!R}
\newtheorem{theorem}{Twierdzenie}[section]

% frontpage
\title{Aplikacja webowa wspomagająca zarządzanie flotą samochodów}
\engtitle{A web application supporting cars fleet management}
\author{Jan Pajdak}
\supervisor{dr inż. Jarosław Mierzwa, K-9}
\guardian{dr hab. inż. Olgierd Unold Prof. nadzw. PWr, K-9}
\field{Informatyka (INF)}
\specialisation{Inżynieria systemów informatycznych (INS)}

\begin{document}
\bibliographystyle{plabbrv}

\maketitle
%\dedication{6cm}{dedykacja \texttt{$\backslash$dedication}}

\tableofcontents 

%----------------------------------------------------------------------------------------
%	SECTION 0
%----------------------------------------------------------------------------------------

%----------------------------------------------------------------------------------------
%	SECTION 1
%----------------------------------------------------------------------------------------
% cel, zakres pracy
\chapter{Wstęp}
Celem niniejszej pracy dyplomowej jest opracowanie projektu, implementacja oraz wdrożenie systemu umożliwiającego zarządzanie flotą samochodów. Pierwszym etapem projektu jest zebrane wymagań funkcjonalnych i niefunkcjonalnych oraz określenie zakresu pracy. Drugi etap projektu to wybór technologii i projekt architektury. Ostatnim celem jest implementacja systemu.

Temat projektu został wybrany ze względu na chęć wykorzystania wiedzy z dziedziny motoryzacji w celu stworzenia aplikacji ułatwiającej zarządzanie pojazdami. Z uwagi na rosnącą popularność rozwiązań związanych z wypożyczaniem samochodów celem projektu jest system, który można opisać jako wewnątrzfirmową wypożyczalnie umożliwiająca jak największe wykorzystanie dostępnej floty pojazdów przez pracowników, którzy nie mają potrzeby posiadania firmowego samochodu na wyłączność. 

W rozdziale pierwszym zawarto wstęp oraz krótki opis celu projektu. Rozdział drugi zawiera opis biznesowy problemu, porównanie z istniejącymi rozwiązaniami oraz wymagania. W kolejnym, trzecim rozdziale znajduje się techniczny opis projektu — wykorzystane technologie i narzędzia oraz architektura systemu.

%----------------------------------------------------------------------------------------
%	SECTION 2
%----------------------------------------------------------------------------------------
\chapter{Specyfikacja problemu}
Celem projektu jest stworzenie aplikacji umożliwiającej wypożyczanie oraz zarządzanie flotą samochodów. Aplikacja jest skierowana do firm które nie mają potrzeby lub wystarczających środków by zapewnić pracownikom samochody na wyłączność. Przykładowym przypadkiem użycia systemu może być jednorazowa potrzeba odwiedzenia klienta lub wyjazd na szkolenie. Typowe rozwiązania dla firm obecne na rynku skierowane są do firm świadczących usługi spedycyjne — aplikacje posiadają warstwę śledzenia ładunków oraz tworzenia zadań przewozowych dla kierowców tocheck orto
%\section{Przegląd istniejących rozwiązań}
\section{Założenia projektowe}
\section{Wymagania funkcjonalne}
\section{Wymagania niefunkcjonalne}

Lorem ipsum dolor sit amet, consectetuer adipiscing elit.

%----------------------------------------------------------------------------------------
%	SECTION 3
%----------------------------------------------------------------------------------------
\chapter{Projekt systemu}
\section{Technologie}
Interfejs użytkownika wykorzystuje platformę \textit{Angular 7}. Podstawowymi elementami w \textit{Angular} są komponenty, każdy z nich złożony z: pliku klasy \textit{TypeScript} zawierającej logikę, wzorca \textit{htm} opisującego wygląd widoku oraz opcjonalnego stylu \textit{css}; w przypadku jego braku styl brany jest z komponentu-rodzica. Warto zwrócić uwagę na język programowania wykorzystywany przez platformę \textit{Angular} — \textit{TypeScript}, będący rozszerzeniem języka \textit{JavaScript}. \textit{TypeScript} dodaje silniejsze typowanie i kładzie większy nacisk na programowanie obiektowe, jednocześnie pozostając w pełni kompatybilnym z \textit{JavaScript}, do którego jest kompilowany i  następnie uruchamiany jest w przeglądarce. Proces kompilacji pozwala na usunięcie wielu błędów, które w przypadku \textit{JavaScript} zostałyby zauważone dopiero po uruchomieniu aplikacji.

Interfejs programistyczny oparty został na technologii \textit{ASP.NET Core 2.1} — jest to nowoczesna platforma oferująca działanie na wielu systemach operacyjnych oraz większa wydajność względem starszych rozwiązań firmy Microsoft. Wykorzystany język programowania to obiektowy, kompilowany i statycznie typowany \textit{C\# 7.3}. Bardzo ważnym elementem tej części projektu jest \textit{EF (Entity Framework) Core 2.1}, framework ORM (Object-Relational Mapping) pozwalający na konwersję miedzy tabelami bazy danych a klasami \textit{C\#}. Jedną z najważniejszych funkcjonalności \textit{EF Core} jest wykorzystana w niniejszym projekcie możliwość utworzenia bazy danych przy użyciu konwencji \textit{Code First}; baza danych jest automatycznie generowana na podstawie klas \text{C\#} znajdujących się w projekcie. 
\section{Narzędzia}
W trakcie realizacji projektu wykorzystane zostały narzędzia najczęściej używane przy wybranych technologiach.

Do zarządzania kodem został wykorzystany system kontroli wersji \textit{Git}. Lokalna kopia projektu była synchronizowana ze zdalnym, prywatnym repozytorium znajdującym się na serwisie \text{GitHub}. Wykorzystane rozwiązanie pozwala na łatwy dostęp do wcześniejszych wersji projektu oraz zmniejsza ryzyko utraty kodu, gdyż nie jest on przechowywany tylko w jednym miejscu.

Ze względu na wykorzystane technologie, kod był rozwijany z pomocą narzędzi Microsoft, oferujących najlepsze wsparcie dla \textit{TypeScript} oraz \textit{C\#}. Aplikacja klienta była rozwijana przy użyciu \textit{Visual Studio Code 1.28}, nowoczesnego edytora który sprawdza się znakomicie przy tworzeniu interfejsów użytkownika ze względu na zintegrowaną konsolę pozwalającą na łatwe zarządzanie paczkami oraz wiele łatwość dostosowywania do potrzeb użytkownika. W trakcie pracy wykorzystano wiele rozszerzeń, najważniejsze z nich to \textit{TSLint}, linter wykrywający błędy w kodzie \textit{TypeScript} oraz \textit{GitLens} — rozszerzenie wspomagające zarządzanie repozytorium \textit{Git}. Do rozwoju serwisów wykorzystano \textit{Visual Studio 2017} pozwalające na łatwe debugowanie kodu oraz analizę aspektów takich jak wykorzystanie zasobów przez program. \textit{Visual Studio} zostało wzbogacone o narzędzie \textit{JetBrains ReSharper} automatycznie formatujące pliki projektu według zadanego wzorca, zapewniając spójność i przejrzystość kodu.

Interfejs programistyczny testowany był przy pomocy \textit{Postman 6.5.2}, aplikacji pozwalającej na wysyłanie oraz zarządzanie zapytaniami HTTP.

%----------------------------------------------------------------------------------------
%	APPENDIX
%----------------------------------------------------------------------------------------
%\appendix
%\chapter{Donec cursus nulla vitae pede}
%Donec cursus nulla vitae pede. Etiam quam pede, aliquet ut, pellentesque sed, sagittis non, est. Quisque egestas malesuada risus. Maecenas ultricies libero a quam. Nullam feugiat arcu. Class aptent taciti sociosqu ad litora torquent per conubia nostra, per inceptos hymenaeos. In interdum, risus ut gravida sollicitudin, leo sapien commodo dui, non consectetuer nisl nunc ac massa. Mauris a orci in eros venenatis euismod. Curabitur orci. Quisque pharetra, dui sed dignissim hendrerit, nibh ante malesuada eros, sed tincidunt magna lorem a tellus. Aliquam erat volutpat. Aenean pulvinar, metus et mattis dictum, massa lacus semper purus, quis vehicula augue mi et leo. Ut eu ipsum. Sed dictum dapibus nisi. Cras mattis.

%----------------------------------------------------------------------------------------
%	END
%----------------------------------------------------------------------------------------
\addcontentsline{toc}{chapter}{Bibliografia} %utworzenie w spisie tre¶ci pozycji Bibliografia
\bibliography{bibliografia} % wstawia bibliografię korzystaj±c z pliku bibliografia.bib - dotyczy BibTeXa, jeżeli nie korzystamy z BibTeXa należy użyć otoczenia thebibliography

%https://angular.io/guide/architecture 3.2
%https://msdn.microsoft.com/en-us/magazine/dn890374.aspx
%https://docs.microsoft.com/en-us/aspnet/core/?view=aspnetcore-2.1

%opcjonalnie może się tu pojawić spis rysunków i tabel
% \listoffigures
% \listoftables
\end{document}

